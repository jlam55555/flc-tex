\section{The Core language}
\label{sec:core}

\subsection{Terminology}
\label{sec:terminology}

\todo{lazy evaluation (call-by-value, call-by-name, call-by-need, applicative-order, normal-order), supercombinators, currying, lambda-abstraction, function application, ADTs (product and sum types) and structured data, whnf, standard representations (booleans, list constructors), standard definitions/prelude (e.g., if, simple combinators, standard representations), scrutinee, definiens, opcodes/instructions, runtime}

\subsection{Syntax}
\label{sec:syntax}

\subsection{Dynamic semantics}
\label{sec:dynamic-semantics}

\todo{function application and basic forms have the same dynamic semantics as the lambda calculus, except using an environment to store variable bindings rather than using substitution}

\subsection{Sample programs}
\label{sec:sample-programs}

\todo{show basic forms}

\todo{twice twice twice}

\todo{show algebraic datatypes}

\subsection{Lexer}
\label{sec:lexer}

\todo{very simple tokenization}

\subsection{Parser}
\label{sec:parser}

\todo{build up LL(1) parser from useful subparsers}

\subsection{Pretty-print utility}
\label{sec:pretty-print}

\todo{constant-time append with indentation, only linear-time rendering}