
\section{Code samples}
\label{sec:code-samples}

\subsection{Core standard library}
\label{sec:core-stdlib}

These are a set of standard definitions that I found useful for exercises. They are not part of the ``prelude,'' i.e., the list of standard definitions builtin to the language as described in \Cref{sec:core-base}, so these definitions need to be loaded from a file.

\inputminted[frame=single]{text}{core/stdlib.core}

\subsection{Project Euler 1}
\label{sec:core-pe1}

See \href{https://projecteuler.net/problem=1}{Project Euler 1}. This requires the standard library from \Cref{sec:core-stdlib}.

\inputminted[frame=single]{text}{core/pe1.core}

\subsection{\texttt{letrec} demonstration}
\label{sec:core-letrec}

The following Core program demonstrates the use of \texttt{letrec}. It also demonstrates a Church encoding of pairs.

\inputminted[frame=single]{text}{core/letrec.core}

\subsection{Infinite streams}
\label{sec:core-streams}

It is easy to generate infinite streams using generator functions. Due to the laziness of Core, these only compute as many elements as are forced.

Alternatively, as in Haskell, it is easy to introduce infinite streams using by implicitly \href{https://wiki.haskell.org/Tying_the_Knot}{``tying the knot,''} due to the simplicity of (mutual) recursion.

\inputminted[frame=single]{text}{core/streams.core}

\subsection{Twice twice twice}
\label{sec:core-twice-twice-twice}

This is a program that has boggled my mind for some time. It illustrates partial application in Core, and has a slightly unusual time complexity. This was inspired by Exercise 2.13 from the tutorial.

\inputminted[frame=single]{text}{core/twice.core}


%%% Local Variables:
%%% mode: latex
%%% TeX-master: "main"
%%% End:
