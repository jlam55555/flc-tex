\documentclass{article}

\newcommand{\todo}[1]{\textbf{TODO: #1}}

\title{
  ECE491 Advanced compilers final report \\
  Implementing a lazy functional language
}
\author{Jonathan Lam}
\date{2022/05/12}

\begin{document}

\maketitle{}

\tableofcontents{}

\section{Project overview}
\label{sec:overview}

\subsection{Motivation}
\label{sec:motivation}

\subsection{Commentary on the tutorial}
\label{sec:tutorial-commentary}

\subsection{Implementation setup}
\label{sec:implementation-setup}

\todo{github repo (all three, including this one)}

\section{Background}
\label{sec:background}

\subsection{Definition of a programming language}
\label{sec:pl-definition}

\todo{syntax, semantics}

\todo{notation used for dynamic semantics here: state machine?}

\subsection{Implementations of programming language}
\label{sec:pl-implementation}

\todo{compilers, interpreters}

\subsection{The untyped $\lambda$-calculus}
\label{sec:ulc}

\subsection{Functional programming}
\label{sec:fp}

\todo{haskell and miranda}

\section{The Core language}
\label{sec:core}

\subsection{Terminology}
\label{sec:terminology}

\todo{lazy evaluation (different terms), supercombinators, etc.}

\subsection{Syntax}
\label{sec:syntax}

\subsection{Dynamic semantics}
\label{sec:dynamic-semantics}

\subsection{Sample programs}
\label{sec:sample-programs}

\subsection{Data structures}
\label{sec:core-data-structures}

\subsection{Lexer}
\label{sec:lexer}

\subsection{Parser}
\label{sec:parser}

\subsection{Pretty-print utility}
\label{sec:pretty-print}

\section{Template instantiation (TI) evaluator}
\label{sec:ti}

\section{G-Machine (GM) compiler}
\label{sec:gm}

\section{Future work}
\label{sec:future-work}

\todo{garbage collection}

\todo{last marks of TI and GM}

\todo{3-address machine and parallel g-machine}

\todo{lambda lifting}

\todo{study implementation of STG}

\section{Conclusions}
\label{sec:conclusions}



\end{document}
