\section{Background}
\label{sec:background}

This section provides brief background on programming language theory from a functional perspective.

\subsection{Definition of a programming language}
\label{sec:pl-definition}

\textit{Interfaces} are necessary for efficient and effective communication. A \textit{programming language} serves as an interface between humans and computers. To rigorously work with a programming language, we define its \textit{syntax} and \textit{semantics}.

The syntax of a programming language describes the way valid expressions and programs are formed. It is specified using a \textit{grammar}. The grammar for the untyped $\lambda$-calculus $\Lambda$ is shown in \Cref{fig:ulc-grammar}.

\begin{figure}
  \centering
  \begin{align*}
    e ::= \lambda x.e \mid x \mid e\ e
  \end{align*}
  \caption{Grammar of $\Lambda$}
  \label{fig:ulc-grammar}
\end{figure}

The \textit{semantics} of a programming language describes its behavior. The \textit{static semantics} denotes the behavior of processes that happen prior to evaluation, such as type checking. The \textit{dynamic semantics} describes the behavior of \textit{evaluation}. Evaluation is the process of reducing an expression down to a \textit{value}, or irreducible expression.

For this project, we actually define two languages: the Core language, and the abstract 

\subsection{Implementations of programming language}
\label{sec:pl-implementation}

\todo{compilers, interpreters}

\subsection{The untyped $\lambda$-calculus}
\label{sec:ulc}

\subsection{Functional programming}
\label{sec:fp}

\todo{haskell and miranda}

